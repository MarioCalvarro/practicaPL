El usuario podrá definir sus propios tipos haciendo un alias de tipos
compuestos («registros» o «arrays»). La estructura para realizar un alias será la
siguiente: la palabra clave \lstinline{incognito} seguido del alias y un igual. Tras esto,
se escribirá la expresión que se abrevia.
\begin{figure}[htbp]
    \centering
    \begin{lstlisting}
//En general
incognito alias = tipo;

//Ejemplos
incognito matriz = ent[][];
incognito estructura = registro {
    a: ent,
    b: bin,
};
    \end{lstlisting}
    \caption{Declaración de un nuevo alias y ejemplos con arrays y registros.}
\end{figure}

\subsection{Conjunto de instrucciones del lenguaje}
En nuestro lenguaje habrá presentes multitud de instrucciones de asignación
dependiendo del tipo de la variable (simples, arrays o registros). Todas ellas
tendrán en común el uso del operador igual (\lstinline{=}) y la siguiente estructura: en el
lado izquierdo de la asignación tendremos una declaración de una variable o su
identificador y en el derecho una expresión con un valor. Las expresiones
posibles son las aritméticas y booleanas habituales (con los operadores
anteriormente definidos), pero también hemos decidido incorporar el operador
ternario (\lstinline{?}) que evalúa una condición y dependiendo de esto asigna un valor u
otro (que vienen dados, a su vez, por expresiones). 
\begin{figure}[htbp]
    \centering
    \begin{lstlisting}
expr_bin? expr_true : expr_false;
    \end{lstlisting}
    \caption{Expresión con ramas condicionales.}
\end{figure}

\subsubsection{Variables simples}
La asignación de variables simples simplemente se realizará con el operador
igual. En el lado izquierdo de la operación se encontrará el identificador de la
variable y en el derecho una expresión del tipo de la variable (aritmética o
lógica):
\begin{figure}[htbp]
    \centering
    \begin{lstlisting}
id = expr;
    \end{lstlisting}
    \caption{Asignación de un valor a una variable.}
\end{figure}
A su vez, será posible combinar las distintas declaraciones vistas anteriormente
con la asignación para declarar y dar valor a la variable en una sola línea.
\begin{figure}[htbp]
    \centering
    \begin{lstlisting}
tipo id = expr_tipo;
tipo id1 = expr_tipo, id2, id3 = expr_tipo...;
    \end{lstlisting}
    \caption{Declaración y asignación de múltiples variables de forma
    simultanea. Cabe destacar que, en este caso, la variable con identificador \lstinline{id2}
    no tiene valor asignado.}
\end{figure}

\subsubsection{Arrays}
Los arrays serán asignados elemento a elemento, separando los valores dados por
comas y todo ello entre corchetes. Para asignar a un array una cantidad de
memoria sin inicializar simplemente escribimos el tipo de los elementos seguido
de, entre corchetes, el número de elementos. De nuevo, podemos declarar una
variable de tipo array y, simultáneamente, asignarle un valor.
\begin{figure}[htbp]
    \centering
    \begin{lstlisting}
//Asignacion a un array la lista de elementos eli
//Los elementos pueden ser, a su vez, arrays
arr = [el1, ..., eln];

//Declaracion y asignacion
tipo arr[] = [el1, ..., eln];

//Reserva de memoria para un array de n dimensiones
tipo arr[]...[] = tipo[i1]...[in] 
    \end{lstlisting}
    \caption{Asignación de valores a los arrays.}
\end{figure}

\subsubsection{Registros}
Por otro lado, los registros se asignarán valor a valor, pero de manera
recursiva, cada campo individualmente y, de nuevo, se podrá declarar y asignar a
la vez. Si simplemente queremos reservar memoria, la asignación se hará de
manera implícita en la declaración.
\begin{figure}[htbp]
    \centering
    \begin{lstlisting}
//Asignacion a un array la lista de elementos eli
//Los elementos pueden ser, a su vez, registros o arrays
reg = {
    atributo11 = expr11,
    atributo12 = expr12,
    ...
    atributo21 = expr21,
    atributo22 = expr22,
    ...
};
    \end{lstlisting}
    \caption{Asignación de valores a los registros.}
\end{figure}

\begin{figure}[htbp]
    \centering
    \begin{lstlisting}
//Declaracion y asignacion
registro {
    atributo11, atributo12, ...: tipo1,
    atributo21, atributo22, ...: tipo2,
    ...
} reg = {
    atributo11 = expr11,
    atributo12 = expr12,
    ...
    atributo21 = expr21,
    atributo22 = expr22,
    ...
};
    \end{lstlisting}
    \caption{Declaración y asignación de valores a los registros.}
\end{figure}

\subsubsection{Ejecución condicional}
La ejecución condicional en este lenguaje tendrá como palabras claves \lstinline{si} y
\lstinline{sino}. Estos condicionales tendrán $n$ ramas siguiendo la siguiente estructura:
empezamos con la palabra clave \lstinline{si} seguida de una expresión condicional que se
evalúa a un «binario» y un bloque anidado de código (que se ejecutará si el
«binario» se evalúa a $1$. Tras esto, le seguirán $n$ bloques que empezarán con
la palabra clave \lstinline{sino} y, opcionalmente, una condición y su correspondiente
bloque anidado de código. Si no existe condición, se interpretará como un «en
caso contrario» y se ejecutará si lo anteriores no lo han hecho.
\begin{figure}[htbp]
    \centering
    \begin{lstlisting}
si expr_cond1 {
    //Codigo1
}
sino expr_cond2 {
    //Codigo2
} 
//Mas ramas condicionales
//...
sino {
    //CodigoN
}
    \end{lstlisting}
    \caption{Ejecución condicional.}
\end{figure}

\subsubsection{Bucles}
Hemos decidido incluir dos tipos de bucles que nombramos por
\lstinline{mientras} y \lstinline{para}
(provenientes de \textit{while} y \textit{for}). La sintaxis que sigue
\lstinline{mientras} es: palabra clave seguida de una expresión condicional y el bloque de
código que se ejecuta mientras se cumpla la condición. Por otro lado, el bucle
\lstinline{para} es: palabra clave, la asignación de una nueva variable (que no es
necesario declarar de forma explícita) a su valor inicial, una flecha y el valor
final (no inclusivo) que tendrá la variable.
\begin{figure}[htbp]
    \centering
    \begin{lstlisting}
mientras expr_cond {
    //Codigo
}

para i = expr_arit_ini -> expr_arit_fin {
    //Codigo
}
    \end{lstlisting}
    \caption{Ejecución iterativa.}
\end{figure}

\subsubsection{Entrada/Salida}
Para realizar la entrada y la salida, el lenguaje contará con una serie de
funciones predefinidas para permitirlo. Estas simplemente permitirán la lectura
de enteros y binarios o su escritura. En caso de que se intente leer o escribir
algo que no coincide con la función llamada, se producirá un error.
\begin{figure}[htbp]
    \centering
    \begin{lstlisting}
leerEnt();
leerBin();
escribirEnt();
escribirBin();
    \end{lstlisting}
    \caption{Funciones de lectura y escritura.}
\end{figure}

\subsubsection{Memoria dinámica}
Para reservar reservar memoria localizada en el \textit{heap} y luego liberarla
hemos decidido incluir dos funciones para estas dos tareas. Para la reserva
tenemos la función \lstinline{reservar} que tiene como argumento el tamaño que
queremos reservar. Debido a que los registros tienen un tamaño heterogéneo,
debemos incluir a su vez una función que devuelva el tamaño de un tipo, que
hemos llamado \lstinline{capacidad} (su argumento es el tipo considerado). Esta
función \lstinline{reservar} devolverá la primera dirección de la memoria
reservada, es decir, se tendrá que asignar a un puntero. A la hora de liberar
esta memoria utilizaremos la función \lstinline{liberar} que tiene como
argumento el puntero que apunta a la memoria reservada.
\begin{figure}[htbp]
    \centering
    \begin{lstlisting}
//Reserva de memoria
tipo @ punt = reservar(capacidad(tipo));

//Liberar memoria
free(punt);
    \end{lstlisting}
    \caption{Funciones de memoria dinámica.}
\end{figure}

\subsubsection{Composición}
Aunque ya se haya dicho de manera implícita, la composición de instrucciones en
este lenguaje se hará mediante el símbolo \lstinline{;}.

\subsection{Expresiones}
La expresiones de nuestro lenguaje se dividirán en aquellas que sean aritméticas
(es decir, que su valor sea un entero) y lógicas (con valor binario). Serán las
siguientes:
\begin{itemize}
    \item \textbf{Constantes}: Serán números enteros habituales (aritméticas) o
        las palabras clave \lstinline{facto} y \lstinline{fake} (lógicas).
    \item \textbf{Variables}: Identificadores de las variables inicializadas
        anteriormente.
    \item \textbf{Operadores infijos}: Expresiones compuestas por dos operandos
        y, entre estos, uno de los anteriormente definidos operadores.
    \item \textbf{Llamadas a funciones}: Siguiendo la definición del operador
        \lstinline{()} definido anteriormente, serán simplemente el
        identificador de la función a llamar y, entre paréntesis, los argumentos
        con los que se la llama.
\end{itemize}
Las expresiones podrán ser, obviamente, una combinación de todas estas y se
podrán entender semánticamente como el valor que producen como resultado.

\subsection{Gestión de errores}
Para la gestión de errores, el compilador simplemente imprimirá el tipo de error
que se ha dado y su localización en el fichero de texto. No habrá recuperación
de errores.
