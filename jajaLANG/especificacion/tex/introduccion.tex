\chapter*{Introducción}
En este documento se presenta una especificación de la sintaxis del lenguaje que
vamos a crear. Hemos decidido llamarlo \textit{jajaLang} debido a que será un
lenguaje de «juguete» (¡y muy divertido de programar!).

Por otra parte, las principales influencias a la hora de decidir como va a ser
la sintaxis han sido \textit{C}, \textit{C++} y \textit{Rust}. Simplemente eran
los lenguajes que mejor manejamos y que en mayor estima tenemos.

A continuación presentamos la especificación detallada del lenguaje, pero
consideramos que antes debemos aclarar un par de cuestiones. En primer lugar,
los ficheros de este lenguaje tendrán la extensión \textit{.jaja}. A su vez, los
comentarios podrán ser de una sola línea (usando //) o de múltiples (usando /*
*/). Todas las sentencias acaban en punto y coma (;).
