\section{Ejemplos}
En esta sección presentaremos ejemplos de algunos programas habituales escritos
en \textit{jajaLang} para ilustrar las características del lenguaje y como
afronta diversas situaciones.

\subsection{Fibonacci}
Programa que calcula los primeros $10$ elementos de la serie de Fibonacci, de forma
recursiva, y los muestra por pantalla (figura~\ref{fig:fibo}).
\begin{figure}[htbp]
    \centering
    \lstinputlisting{../src/test/resources/fibonacci.jaja}
    \caption{Programa que calcula la serie de Fibonacci e importa otro fichero.}
    \label{fig:fibo}
\end{figure}

\subsection{Lectura/Escritura}
Programa que lee un entero y un binario. Si el binario es \lstinline{facto}, se
imprimirá el entero. En caso contrario, no se hará nada
(figura~\ref{fig:lecesc}).
\begin{figure}[htbp]
    \centering
    \lstinputlisting{../src/test/resources/lecturaEscritura.jaja}
    \caption{Programa que lee un número y una condición que, si se cumple, hace
    que se imprima el entero.}
    \label{fig:lecesc}
\end{figure}

\subsection{Par/Impar}
Programa que calcula si un número es par (\lstinline{facto}) o impar
(\lstinline{fake}). Para ello se hace uso de una recursión mutua.
(figura~\ref{fig:parimpar})
\begin{figure}[htbp]
    \centering
    \lstinputlisting{../src/test/resources/parImpar.jaja}
    \caption{Programa determina si un número es par o impar, usando recursión
    mutua entre dos funciones. No funciona en porque no tenemos implementada la recursión mutua.}
    \label{fig:parimpar}
\end{figure}

\subsection{Registros y arrays}
Programa que comprueba el funcionamiento de los registros y los arrays, el
acceso a sus atributos/elementos y su modificación (figura~\ref{fig:regarray}).
\begin{figure}[htbp]
    \centering
    \lstinputlisting{../src/test/resources/registrosArrays.jaja}
    \caption{Programa que trata con los atributos/elementos de los
    registros/arrays. Además, hace uso de las asignaciones/operadores.}
    \label{fig:regarray}
\end{figure}

\subsection{Paso por referencia y memoria dinámica}
Programa que pasa por referencia una variable y utiliza memoria dinámica
(figura~\ref{fig:refdin}).
\begin{figure}[htbp]
    \centering
    \lstinputlisting{../src/test/resources/referenciaDinamica.jaja}
    \caption{Programa que contiene una función con un parámetro que se pasa por
    referencia y punteros que hacen uso de memoria dinámica.}
    \label{fig:refdin}
\end{figure}
