\section{Ejemplos}
En esta sección presentaremos ejemplos de algunos programas habituales escritos
en \textit{jajaLang} para ilustrar las características del lenguaje y como
afronta diversas situaciones.

\subsection{Fibonacci}
Programa que calcula los primeros $10$ elementos de la serie de Fibonacci, de forma
recursiva, y los muestra por pantalla.
\begin{figure}[htbp]
    \centering
    \begin{lstlisting}
diver fibonacci(r: ent) -> ent {
    si c == 0 {
        devuelve 1; 
    }
    si c == 1 {
        devuelve 1;
    }
    devuelve fibonacci(c-1) + fibonacci(c-2);
}

diver tronco() {
    para i = 0 -> 10 {
        escribirEntero(fibonacci(i));
    } 
}
    \end{lstlisting}
    \caption{Programa que calcula la serie de Fibonacci.}
\end{figure}

\subsection{Lectura/Escritura}
Programa que lee un entero y un binario. Si el binario es \lstinline{facto}, se
imprimirá el entero. En caso contrario, no se hará nada.
\begin{figure}[htbp]
    \centering
    \begin{lstlisting}
diver tronco() {
    ent num = leerEnt();
    bin cond = leerBin();
    si cond {
        escribirEnt(num);
    }
}
    \end{lstlisting}
    \caption{Programa que lee un número y una condición que, si se cumple, hace
    que se imprima el entero.}
\end{figure}

\subsection{Par/Impar}
Programa que calcula si un número es par (\lstinline{facto}) o impar
(\lstinline{fake}). Para ello se hace uso de una recursión mutua.
\begin{figure}[htbp]
    \centering
    \begin{lstlisting}
diver par(x: ent) -> bin {
    si x == 0 {
        devuelve facto;
    }
    sino x == 1 {
        devuelve fake;
    }
    sino {
        devuelve impar(x-1);
    }
}

diver impar(x: ent) -> bin {
    si x == 0 {
        devuelve fake;
    }
    sino x == 1 {
        devuelve facto;
    }
    sino {
        devuelve par(x-1);
    }
}

diver tronco() {
    escribeBin(impar(9));
    escribeBin(par(12));
}
    \end{lstlisting}
    \caption{Programa determina si un número es par o impar, usando recursión
    mutua entre dos funciones.}
\end{figure}

\subsection{Registros y arrays}
Programa que comprueba el funcionamiento de los registros y los arrays, el
acceso a sus atributos/elementos y su modificación.
\begin{figure}[htbp]
    \centering
    \begin{lstlisting}
incognito dato = registro { arr: vector(ent, 3), };

diver tronco() {
    dato x = nuevo { arr = nuevo [1, 2, 3], a = 3 };
    escribeEnt(x.arr[0]);
    x.arr[0] = 9;       //Asociativo por la izquierda
    escribeEnt(x.arr[0]);
}
    \end{lstlisting}
    \caption{Programa que trata con los atributos/elementos de los
    registros/arrays.}
\end{figure}

\subsection{Paso por referencia y memoria dinámica}
Programa que pasa por referencia una variable y utiliza memoria dinámica.
\begin{figure}[htbp]
    \centering
    \begin{lstlisting}
incognito dato_t = registro {
    atr1: ent @,
    atr2: vector(ent @, 2),
};

diver f(&a: ent) -> ent {
    devuelve @a;
}

diver tronco() {
    ent @ punt = nuevo ent;
    @punt = 10;
    ent b = f(punt);
    escribeEnt(b); //Tiene que dar 10
    liberar(punt);


    //Ejemplo complicado de punteros
    //registros y arrays con mem. dinamica
    dato_t @ d = nuevo dato_t;
    @d = nuevo {
        atr1 = nuevo ent,
        atr2 = nuevo [nuevo ent, nuevo ent]
    };
    @(@d).atr1 = 4;
    @(@d).atr2[0] = 1;
    @(@d).atr2[1] = 2;
}
    \end{lstlisting}
    \caption{Programa que contiene una función con un parámetro que se pasa por
    referencia y punteros que hacen uso de memoria dinámica.}
\end{figure}
