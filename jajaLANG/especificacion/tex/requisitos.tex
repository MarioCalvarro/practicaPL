\section{Requisitos}
En esa sección daremos los detalles sobre la sintaxis de nuestro lenguaje.
Siguiendo de manera aproximada el enunciado dividiremos las partes de este
apartado.

\subsection{Léxico}
Veamos en primer lugar el léxico, es decir, los \textit{tokens} que manejaremos
en nuestro lenguaje. Únicamente se manejan caracteres ASCII:
\subsubsection{Palabras clave}
Listado de las palabras clave del lenguaje:
\begin{itemize}
    \item \lstinline{ent}
    \item \lstinline{bin}
    \item \lstinline{facto}
    \item \lstinline{fake}
    \item \lstinline{si}
    \item \lstinline{sino}
    \item \lstinline{mientras}
    \item \lstinline{para}
    \item \lstinline{diver}
    \item \lstinline{registro}
    \item \lstinline{nulo}
    \item \lstinline{incognito}
    \item \lstinline{devuelve}
    \item \lstinline{#traficar}
    \item \lstinline{como}
    \item \lstinline{vector}
    \item \lstinline{nuevo}
\end{itemize}

\subsubsection{Identificadores}
Los identificadores de las variables serán una letra o un guion bajo seguido de
una secuencia de letras y números.

\subsubsection{Espacios}
Los espacios o secuencia nulas en nuestro lenguaje serán los espacio propiamente
dichos (` '), saltos de línea (`\lstinline{\n}'), tabuladores
(`\lstinline{\t}`), \textit{carriage return} (`\lstinline{\r}') y el
\textit{backspace} (`\lstinline{\b}').

\subsubsection{Comentarios}
Los comentarios en este lenguaje podrán ser de una sola línea (\lstinline{//}) o
ocupar múltiples (se abren con \lstinline{/*} y se cierran con \lstinline{*/}).

\subsection{Identificadores y ámbitos de definición}
Veamos la declaración de los distintos tipos de variables.
\subsubsection{Variables simples}
A la hora de declarar una nueva variable debemos indicar el tipo de la misma y
su identificador.
\begin{figure}[htbp]
    \centering
    \begin{lstlisting}
tipo id;
    \end{lstlisting}
    \caption{Declaración de una variable simple.}
\end{figure}
También es posible declarar múltiples variables del mismo tipo, separándolas por
comas.
\begin{figure}[htbp]
    \centering
    \begin{lstlisting}
tipo id1, id2...;
    \end{lstlisting}
    \caption{Declaración de múltiples variables.}
\end{figure}


\subsubsection{Arrays}
Respecto a los arrays, permitiremos crear arrays de un tipo genérico, pero
tamaño dado en tipo de compilación mediante la palabra clave \lstinline{vector}.
Con esto permitiremos crear arrays de un número genérico de dimensiones
(simplemente en tipo ponemos otro array).
\begin{figure}[htbp]
    \centering
    \begin{lstlisting}
vector(tipo, tam) arr;
    \end{lstlisting}
    \caption{Declaración de arrays.}
\end{figure}

\subsubsection{Registros y punteros}
Hemos decidido incluir en nuestro lenguaje tanto punteros como registros. La
declaración de los punteros la realizaremos similarmente a la de las variables,
pero incluyendo el símbolo (\lstinline{@}) entre el tipo y el identificador. Respecto a los
registros se hace de manera similar a los arrays. Primero, utilizamos la palabra clave \lstinline{registro}. Tras esto, iniciamos un bloque anidado
con los campos del registro, separados por comas (el último también
puede tener coma al final de forma opcional), y, separado por punto y coma, su tipo.
\begin{figure}[htbp]
    \centering
    \begin{lstlisting}
//Registros
registro {
    atributo11, atributo12, ...: tipo1,
    atributo21, atributo22, ...: tipo2,
    ...
} id1, id2, ...;

//Punteros
tipo @ id;
    \end{lstlisting}
    \caption{Declaración de variables registro y de punteros.}
\end{figure}

\subsubsection{Bloques anidados}
Los bloques anidados simplemente serán delimitados por llaves.
\begin{figure}[htbp]
    \centering
    \begin{lstlisting}
{
    {
        //Codigo
    }
    {
        //Codigo
    }
}
    \end{lstlisting}
    \caption{Bloque con otros dos bloques anidados.}
\end{figure}

\subsubsection{Funciones}
Las funciones se componen de cinco partes diferenciadas. Primero, declaramos que
es una función a través de la palabra clave \lstinline{diver} (que proviene de
la palabra inglesa \textit{fun} (\textit{function})). Tras esto, incluimos el
nombre que se le da a la función seguido de los argumentos, separados por
flechas. Estos argumentos tendrán la siguiente forma: identificador del
parámetro y su tipo. Por defecto, el paso de parámetros será por valor, en caso
de que sea por referencia, se deberá añadir el símbolo \lstinline{&} antes del
identificador. Finalmente, a través de una flecha indicamos el tipo de retorno
de la función y un bloque anidado con el cuerpo de la función. El tipo de
retorno es opcional en el caso de que únicamente se modifique el estado del
programa. En el caso de que queramos devolver un valor, utilizaremos la
sentencia \lstinline{devuelve valor;} que, además, hará que la ejecución del
programa vuelva al lugar desde el que se llamó a la función.
\begin{figure}[htbp]
    \centering
    \begin{lstlisting}
diver id1 (par1: tipo1 -> par2: tipo2 -> ... ) {
    //Codigo
}
diver id2 (&par1: tipo1 -> par2: tipo2 -> ... ) -> tipo {
    //Codigo
    //...
    devuelve valor;
}
    \end{lstlisting}
    \caption{Declaración de funciones con y sin tipo de retorno y con parámetros
    por valor y por referencia.}
\end{figure}

\subsubsection{Importación de código}
Para importar código procedente de otros ficheros utilizamos la palabra clave
\lstinline{#traficar} seguido de la localización del otro fichero. Tras esto,
añadimos la palabra clave \lstinline{como} y el alias del \textit{namespace} del
fichero importado (para evitar la colisión de nombres).
\begin{figure}[htbp]
    \centering
    \begin{lstlisting}
#traficar ruta/a/fichero.jaja como namespace_id
    \end{lstlisting}
    \caption{Importación de código procedente de otro fichero.}
\end{figure}
Para llamar a las funciones y acceder a las variables declaradas en este otro
fichero, usaremos el operador \lstinline{::} precedido del identificador del
\textit{namespace} y seguido de aquello a lo que queremos acceder.

\subsection{Tipos}
Como ya hemos indicado anteriormente, las variables tienen que venir declaradas
de forma explícita y su tipado es estático. Los tipos predefinidos del lenguaje
serán los «enteros» de $32$ bits (que declararemos con la palabra clave
\lstinline{ent}) y los «binarios» (booleanos, con palabra clave
\lstinline{bin}). La lista de operadores predefinidos será la siguiente.
\begin{itemize}
\item Operadores aritméticos:
    \begin{enumerate}
        \item Potenciación: \lstinline{^}.
        \item Producto (\lstinline{*}), división (\lstinline{/}) y módulo (\lstinline{%}).
        \item Suma (\lstinline{+}) y resta (\lstinline{-}).
    \end{enumerate}
\item Operadores relacionales: \lstinline{==}, \lstinline{!=}, \lstinline{>}, \lstinline{<},
    \lstinline{>=}, \lstinline{<=}.

\item Operadores lógicos:
\begin{enumerate}
    \item Negación lógica: \lstinline{!}
    \item «Y» lógico: \lstinline{&&}
    \item «O» inclusivo lógico: \lstinline{||}
\end{enumerate}

\item Otros operadores:
\begin{itemize}
    \item Acceso a atributos de un registro: \lstinline{.}
    \item Acceso a un elemento de un array: \lstinline{[]}
    \item Llamada a una función (identificador de la función a la izquierda,
        parámetros dentro): \lstinline{()}
    \item Opuesto de un entero: \lstinline{-}
    \item Acceso a la dirección de una variable: \lstinline{&}
    \item Acceso al valor apuntado por un puntero: \lstinline{@}
\end{itemize}
Por último, la asignación será \lstinline{=} y se podrá combinar con las
operaciones aritméticas y lógicas.
\end{itemize}
\begin{table}[H]
    \centering
    \begin{tabular}{ | c | c | c | } 
        \hline
        Operador & Prioridad & Asociatividad \\ 
        \hline
        \lstinline$()$ & $0$ & Izquierda \\ 
        \hline
        \lstinline$[] .$ & $1$ & Izquierda \\ 
        \hline
        \lstinline$-$ unario & $2$ & Derecha \\ 
        \hline
        \lstinline$& @$ & $3$ & - \\ 
        \hline
        \lstinline$* / %$ & $4$ & Izquierda \\ 
        \hline
        \lstinline$+ -$ & $5$ & Izquierda \\ 
        \hline
        \lstinline$ < > <= >=$ & $6$ & Izquierda \\ 
        \hline
        \lstinline$== !=$ & $7$ & Izquierda \\ 
        \hline
        \lstinline$&&$ & $8$ & Izquierda \\ 
        \hline
        \lstinline$||$ & $9$ & Izquierda \\ 
        \hline
        \lstinline$= += ...$ & $10$ & Derecha \\ 
        \hline
    \end{tabular}
    \caption{Tabla con los distintos operadores, su prioridad y su
    asociatividad.}
\end{table}
