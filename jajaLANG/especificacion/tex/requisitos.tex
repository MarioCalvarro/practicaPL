\chapter*{Requisitos}
En esa sección daremos los detalles sobre la sintaxis de nuestro lenguaje.
Siguiendo de manera aproximada el enunciado dividiremos las secciones de este
apartado.

\section*{Léxico}
Veamos en primer lugar el léxico, es decir, los \textit{tokens} que manejaremos
en nuestro lenguaje. Únicamente se manejan caracteres ASCII:
\subsection*{Palabras clave}
Listado de las palabras clave del lenguaje:
\begin{itemize}
    \item \lstinline{ent}
    \item \lstinline{bin}
    \item \lstinline{true}
    \item \lstinline{false}
    \item \lstinline{si}
    \item \lstinline{mientras}
    \item \lstinline{diver}
    \item \lstinline{registro}
    \item \lstinline{incognito}
\end{itemize}

\subsection*{Identificadores}
Los identificadores de las variables serán una letra o un guion bajo seguido de
una secuencia de letras y números.

\subsection*{Espacios}
Los espacios o secuencia nulas en nuestro lenguaje serán los espacio propiamente
dichos (` '), saltos de línea (`\lstinline{\n}'), tabuladores
(`\lstinline{\t}`), \textit{carriage return} (`\lstinline{\r}') y el
\textit{backspace} (`\lstinline{\b}').

\section*{Identificadores y ámbitos de definición}
Veamos la declaración de los distintos tipos de variables.
\subsection*{Variables simples}
A la hora de declarar una nueva variable debemos indicar el tipo de la misma y
su identificador.
\begin{figure}[H]
    \centering
    \begin{lstlisting}
tipo id;
    \end{lstlisting}
    \caption{Declaración de una variable simple.}
\end{figure}
También es posible declarar múltiples variables del mismo tipo, separándolas por
comas:
\begin{figure}[H]
    \centering
    \begin{lstlisting}
tipo id1, id2;
    \end{lstlisting}
    \caption{Declaración de múltiples variables.}
\end{figure}


\subsection*{Arrays}
Respecto a los arrays, permitiremos la creación de arrays de dimensión
arbitraria cuyo tamaño puede venir dado en tiempo de ejecución por una variable
o en compilación de forma constante. La sintaxis para declarar estos arrays será
el tipo de los elementos del array, seguido del identificador del mismo y
corchetes que abren y cierran. Cada pareja de corchetes indicará una dimensión
más. Por ejemplo,
\begin{figure}[H]
    \centering
    \begin{lstlisting}
tipo arr[]...[];
    \end{lstlisting}
    \caption{Declaración de arrays de dimensión arbitraria.}
\end{figure}

\subsection*{Registros y punteros}
Hemos decidido incluir en nuestro lenguaje tanto punteros como registros. La
declaración de los punteros la realizaremos similarmente a la de las variables,
pero incluyendo el símbolo (\lstinline{@}) entre el tipo y el identificador. Respecto a los
registros se hace de manera similar: primero, utilizamos la palabra clave \lstinline{registro}. Tras esto, iniciamos un bloque anidado
con los campos del registro y su tipo, separados por comas (el último también
puede tener coma al final de forma opcional).
\begin{figure}[H]
    \centering
    \begin{lstlisting}
//Registros
registro {
    atributo11, atributo12, ...: tipo1,
    atributo21, atributo22, ...: tipo2,
    ...
} id1, id2, ...;

//Punteros
tipo @ id;
    \end{lstlisting}
    \caption{Declaración de variables registro y de punteros.}
\end{figure}

\subsection*{Bloques anidados}
Los bloques anidados simplemente serán delimitados por llaves.
\begin{figure}[H]
    \centering
    \begin{lstlisting}
{
    {
        //Codigo
    }
    {
        //Codigo
    }
}
    \end{lstlisting}
    \caption{Bloque con otros dos bloques anidados.}
\end{figure}

\subsection*{Funciones}
Las funciones se componen de cinco partes diferenciadas. Primero, declaramos que
es una función a través de la palabra clave \lstinline{diver} (que proviene de la palabra
inglesa \textit{fun} (\textit{function})). Tras esto, incluimos el nombre que se
le da a la función seguido de los argumentos, separados por flechas. Estos
argumentos tendrán la siguiente forma: identificador del parámetro y su tipo.
Por defecto, el paso de parámetros será por valor, en caso de que sea por
referencia, se deberá añadir el símbolo \lstinline{&} antes del identificador.
Finalmente, a través de una flecha indicamos el tipo de retorno de la función y
un bloque anidado con el cuerpo de la función. El tipo de retorno es opcional en
el caso de que únicamente se modifique el estado del programa. Por ejemplo,
\begin{figure}[H]
    \centering
    \begin{lstlisting}
diver id1 (par1: tipo1 -> par2: tipo2 -> ... ) {
    //Codigo
}
diver id2 (&par1: tipo1 -> par2: tipo2 -> ... ) -> tipo {
    //Codigo
}
    \end{lstlisting}
    \caption{Declaración de funciones con y sin tipo de retorno y con parámetros
    por valor y por referencia.}
\end{figure}

\subsection*{Importación de código}
Para importar código procedente de otros ficheros utilizamos la palabra clave
\lstinline{#traficar} seguido de la localización del otro fichero. Por ejemplo,
\begin{figure}[H]
    \centering
    \begin{lstlisting}
#traficar ruta/a/fichero.jaja
    \end{lstlisting}
    \caption{Importación de código procedente de otro fichero.}
\end{figure}

\section*{Tipos}
Como ya hemos indicado anteriormente, las variables tienen que venir declaradas
de forma explícita y su tipado es estático. Los tipos predefinidos del lenguaje
serán los «enteros» de $32$ bits (que declararemos con la palabra clave
\lstinline{ent}) y los «binarios» (booleanos, con palabra clave
\lstinline{bin}). La lista de operadores predefinidos será la siguiente.
\begin{itemize}
\item Operadores aritméticos:
    \begin{enumerate}
        \item Potenciación: \lstinline{^}.
        \item Producto (\lstinline{*}), división (\lstinline{/}) y módulo (\lstinline{%}).
        \item Suma (\lstinline{+}) y división (\lstinline{-}).
    \end{enumerate}
\item Operadores relacionales: \lstinline{==}, \lstinline{!=}, \lstinline{>}, \lstinline{<},
    \lstinline{>=}, \lstinline{<=}.

\item Operadores lógicos:
\begin{enumerate}
    \item Negación lógica: \lstinline{!}
    \item «Y» lógico: \lstinline{&&}
    \item «O» inclusivo lógico: \lstinline{||}
\end{enumerate}

\item Otros operadores:
\begin{itemize}
    \item Acceso a atributos de un registro: \lstinline{.}
    \item Acceso a un elemento de un array: \lstinline{[]}
    \item Llamada a una función (identificador de la función a la izquierda,
        parámetros dentro): \lstinline{()}
    \item Opuesto de un entero: \lstinline{-}
    \item Acceso a la dirección de una variable: \lstinline{&}
    \item Acceso al valor apuntado por un puntero: \lstinline{@}
\end{itemize}
Por último, la asignación será \lstinline{=} y se podrá combinar con las
operaciones aritméticas y lógicas.
\end{itemize}
\begin{table}[H]
    \centering
    \begin{tabular}{ | c | c | c | } 
        \hline
        Operador & Prioridad & Asociatividad \\ 
        \hline
        \lstinline$[] .$ & $0$ & Izquierda \\ 
        \hline
        \lstinline$()$ & $1$ & Izquierda \\ 
        \hline
        \lstinline$-$ unario & $2$ & Derecha \\ 
        \hline
        \lstinline$& @$ & $3$ & - \\ 
        \hline
        \lstinline$* / %$ & $4$ & Izquierda \\ 
        \hline
        \lstinline$+ -$ & $5$ & Izquierda \\ 
        \hline
        \lstinline$ < > <= >=$ & $6$ & Izquierda \\ 
        \hline
        \lstinline$== !=$ & $7$ & Izquierda \\ 
        \hline
        \lstinline$&&$ & $8$ & Izquierda \\ 
        \hline
        \lstinline$||$ & $9$ & Izquierda \\ 
        \hline
        \lstinline$= += ...$ & $10$ & Derecha \\ 
        \hline
    \end{tabular}
    \caption{Tabla con los distintos operadores, su prioridad y su
    asociatividad.}
\end{table}
